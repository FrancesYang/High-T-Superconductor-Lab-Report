\documentclass[prb,preprint]{revtex4-1} 
% The line above defines the type of LaTeX document.
% Note that AJP uses the same style as Phys. Rev. B (prb).

% The % character begins a comment, which continues to the end of the line.

\usepackage{amsmath}  % needed for \tfrac, \bmatrix, etc.
\usepackage{amsfonts} % needed for bold Greek, Fraktur, and blackboard bold
\usepackage{graphicx} % needed for figures

\begin{document}

% Be sure to use the \title, \author, \affiliation, and \abstract macros
% to format your title page.  Don't use lower-level macros to  manually
% adjust the fonts and centering.

\title{Critical Temperature of YBCO and BSCCO Superconductors}
% In a long title you can use \\ to force a line break at a certain location.

\author{}
\affiliation{}

% optional second address
% If there were a second author at the same address, we would put another 
% \author{} statement here.  Don't combine multiple authors in a single
% \author statement.


\author{Frances Yang}
\affiliation{Department of Physics, Smith College, Northampton, MA 01063}

% See the REVTeX documentation for more examples of author and affiliation lists.

\date{\today}

\begin{abstract}
This article explains and illustrates the use of \LaTeX\ in preparing manuscripts
for submission to the American Journal of Physics (AJP). While it is not a
comprehensive reference, we hope it will suffice for the needs of most
AJP authors.
\end{abstract}
% AJP requires an abstract for all regular article submissions.
% Abstracts are optional for submissions to the "Notes and Discussions" section.

\maketitle % title page is now complete


\section{Introduction} % Section titles are automatically converted to all-caps.
% Section numbering is automatic.
In 1911, Kamerlingh Onnes discovered superconductivity in mercury.  
Instead of a gradual decrease in the electrical resistance as the temperature decreased, he found a sudden loss of resistance when the temperature went below 4.2 K. 
The temperature at which the resistance disappears is called the critical temperature.
In addition to having no resistance, superconductors also exhibit perfect diamagnetism, which means they exclude all magnetic field flux from their interior. 
This property, called the Meissner effect, was discovered in 1933.\cite{intro}
A common demonstration of the Meissner effect is to place a a magnet on top of a superconductor above its critical temperature. When the superconductor is cooled to become superconducting, the magnet will start to levitate above the superconductor. 
This phenomenon can be used to make levitation trains.  Superconductors are also used to create magnetic fields used in magnetic resonance imaging and in high energy physics to accelerate and control the path of particles.\cite{kumar}

Superconductivity has been found to occur in elements, alloys, binary compounds, and other materials.  
Roughly 70 years after superconductivity was discovered, ceramic materials with critical temperatures above the liquid nitrogen temperature of 77 K were found.\cite{melissinos} 
These ``high temperature'' superconductors were made up of alternating layers of rare earth atoms and copper and oxygen atoms.\cite{kumar} 
In our experiment, we look at two high temperature superconductors, $\textrm{Yt}\textrm{Ba}_{2}\textrm{Cu}_{3}\textrm{O}_{7-\delta}$ and $\textrm{Bi}_{2}\textrm{Sr}_{2}\textrm{Ca}_{n-1}\textrm{Cu}_{n}\textrm{O}_{9}$. %n=3 mainly for colorado superconductor 110 Tc
 

 
\section{Aims}
\begin{enumerate}
\item To determine the critical temperature of YBCO and BSCCO.
\end{enumerate}

\section{Procedure}
The YBCO and BSCCO superconductors were obtained from Colorado Superconductor Inc. Each superconductor  is contained in a brass casing with three pairs of leads. 
One pair was for attaching to the current source, another pair was used to measure voltage across the superconductor, and the last pair that of the thermocouple inside the casing. 
The thermocouple has a resistance that varies with temperature, so we can determine the  temperature of superconductor by measuring the voltage across the thermocouple. 

We used different methods to measure the critical temperature of the two superconductors.
For the YBCO, we connected it to a source with an alternating current of $\pm$1 mA. 
The voltage across the YBCO sample, $V_{\textrm{YBCO}}$, and the voltage across the thermocouple, $V_{T}$ were measured with two nanovoltmeters. We placed the superconductor inside a styrofoam cup filled with sand. 
The sand reduces the rate at which the superconductor heats up, which allows us to gather more data points. 
We cooled the YBCO sample by pouring liquid nitrogen into the styrofoam cup. We waited for the superconductor to settle to a minimum temperature. Then we recorded $V_{\textrm{YBCO}}$ and $V_{T}$ as it warmed up to above its critical temperature. 

We used a two-phase lock-in amplifier for the BSCCO experiment. In an AC circuit, the measurement of resistance includes contributions from capacitors and inductors. 
Although there are no explicit capacitors or inductors in our circuit, they are effectively present due to the electrical cables used. 
The lock-in amplifier allows us to extract the resistance of superconductor from the other contributions because the resistance will be in phase with the signal, while the contributions of capacitors and inductors will be out of phase. 
Since the resistance of the BSCCO at room temperature is on the order of m$\Omega$s, we can get approximately $\pm$1 mA in the circuit by connecting the BSCCO in series with a 1 k$\Omega$ resistor, and with a AC voltage source of 1 V. 



\section{Results and Analysis}

\section{Discussion}

\section{Conclusion}



\begin{thebibliography}{99}
% The numeral (here 99) in curly braces is nominally the number of entries in
% the bibliography. It's supposed to affect the amount of space around the
% numerical labels, so only the number of digits should matter--and even that
% seems to make no discernible difference.

\bibitem{intro} Charles Poole, Horacio Farach, and Richard Creswick, \textit{Superconductivity} (Academic Press, 2007).

\bibitem{kumar} Ajay Kumar Saxena, \textit{High-Temperature Superconductors} (Springer Berlin Heidelberg, 2010).

\bibitem{melissinos} Experiments in Modern Physics.  Adrian Constantin Melissinos and Jim Napolitano.  Gulf Professional Publishing, 2003



\end{thebibliography}

% If your manuscript is conditionally accepted, the editors will ask you to
% submit your editable LaTeX source file.  Before doing so, you should move
% all tables and figure captions to the end, as shown below.  Tables come 
% first, followed by figure captions (with figure inclusions commented-out).
% Figures should be submitted as separate files, collected with the
% LaTeX file into a single .zip archive.

%\newpage   % Start a new page for tables

%\begin{table}[h!]
%\centering
%\caption{Elementary bosons}
%\begin{ruledtabular}
%\begin{tabular}{l c c c c p{5cm}}
%Name & Symbol & Mass (GeV/$c^2$) & Spin & Discovered & Interacts with \\
%\hline
%Photon & $\gamma$ & \ \ 0 & 1 & 1905 & Electrically charged particles \\
%Gluons & $g$ & \ \ 0 & 1 & 1978 & Strongly interacting particles (quarks and gluons) \\
%Weak charged bosons & $W^\pm$ & \ 82 & 1 & 1983 & Quarks, leptons, $W^\pm$, $Z^0$, $\gamma$ \\
%Weak neutral boson & $Z^0$ & \ 91 & 1 & 1983 & Quarks, leptons, $W^\pm$, $Z^0$ \\
%Higgs boson & $H$ & 126 & 0 & 2012 & Massive particles (according to theory) \\
%\end{tabular}
%\end{ruledtabular}
%\label{bosons}
%\end{table}

%\newpage   % Start a new page for figure captions

%\section*{Figure captions}

%\begin{figure}[h!]
%\centering
%\includegraphics{GasBulbData.eps}   % This line stays commented-out
%\caption{Pressure as a function of temperature for a fixed volume of air.  
%The three data sets are for three different amounts of air in the container. 
%For an ideal gas, the pressure would go to zero at $-273^\circ$C.  (Notice
%that this is a vector graphic, so it can be viewed at any scale without
%seeing pixels.)}

%\label{gasbulbdata}
%\end{figure}

%\begin{figure}[h!]
%\centering
%\includegraphics[width=5in]{ThreeSunsets.jpg}   % This line stays commented-out
%\caption{Three overlaid sequences of photos of the setting sun, taken
%near the December solstice (left), September equinox (center), and
%June solstice (right), all from the same location at 41$^\circ$ north
%latitude. The time interval between images in each sequence is approximately
%four minutes.}
%\label{sunsets}
%\end{figure}

\end{document}
